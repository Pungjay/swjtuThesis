%-------------------------------------------------------%
%                        第2章内容
%-------------------------------------------------------%
\chapter{格式规范}\label{第2章}

\section{语言和表述}

研究生学位论文要求用汉语书写,所用汉字须符合教育部、国家语言文字工作委员会组织制定的《通用规范汉字表》。专用名词、术语可采用国际通用的代号,量及其单位所使用的符号应符合国家标准《国际单位制及其应用》(GB3100—1993)、《有关量、单位和符号的一般原则》(GB/T 3101—1993)的规定。图、表中的图题、坐标轴、图例、表头等描述性的词组或语句须使用汉语,专用名词术语、物理量及其单位可使用符合规范要求的符号。

根据教育部、外交部、公安部联合制定的《学校招收和培养国际学生管理办法》(教育部令第42号)有关规定,使用外国语言接受高等教育学历的国际学生,其学位论文可使用相应的外文撰写,但须附\textcolor{red}{5000}字的详细中文摘要,并放在英文摘要之前。

学位论文表述要严谨简明,重点突出,专业常识应简写或不写,做到立论正确、层次分明、数据可靠、文字凝练、说理透彻、推理严谨,避免使用文学性质的带感情色彩的非学术性词语。\textbf{学位论文作者具有唯一性,避免“我们”等用词}。

\section{标题和层次}

论文各章节标题要突出重点、简明扼要,尽量控制在一行,标题中不加标点符号。标题中尽量不采用英文缩写词,必须采用时应使用本行业的通用缩写词。

论文章节层次要清楚,\textbf{一般到三级层级(例如“1.1.1”)即可},最多到四级层次。各章节层次均应有标题,标题由序号和名称组成,之间空1个汉字符。一级标题(章标题)居中书写,次级标题顶格书写。

\section{字体和段落}

\textbf{若无特殊说明,论文中的中文统一用宋体,数字和英文统一用Times New Roman字体。从中文摘要开始,所有文字段落和标题行间距均取固定值20磅;所有段落按两端对齐、首行缩进2个汉字符方式书写内容。}

\textbf{中文用黑体或加粗的地方,对应数字和英文宜使用加粗Times New Roman字体。}

中、英文字号对应关系如表\ref{tab2-1}所示,主要文字及段落格式要求如表\ref{tab2-2}所示。


\begin{table}[!ht]
\centering
\caption{中、英文字号对应关系}
\label{tab2-1}
\begin{tabularx}{\textwidth}{ 
>{\centering\arraybackslash}X 
>{\centering\arraybackslash}X
>{\centering\arraybackslash}X
>{\centering\arraybackslash}X
}
\toprule
\textbf{中文字号} & \textbf{英文磅数} & \textbf{中文字号} & \textbf{英文磅数} \\
\midrule
二号 & 22 & 四号 & 14 \\
小二 & 18 & 小四 & 12 \\
三号 & 16 & 五号 & 10.5 \\
小三 & 15 & 小五 & 9 \\
\bottomrule
\end{tabularx}
\end{table}

\begin{table}[!ht]
\centering
\caption{主要文字及段落格式要求}
\label{tab2-2}
\begin{tabular}{ccccccm{0.3\textwidth}}
\toprule
\textbf{内容} & \textbf{字体} & \textbf{字号} & \textbf{对齐方式} & \textbf{段前距} & \textbf{段后距} & \textbf{备注} \\
\midrule
一级标题 & 黑体 & 三号 & 居中 & 24磅 & 18磅 &  \\
二级标题 & 黑体 & 四号 & 顶格左对齐 & 18磅 & 6磅 &  \\
三级标题 & 黑体 & 四号 & 顶格左对齐 & 12磅 & 6磅 &  \\
四级标题 & 黑体 & 小四 & 顶格左对齐 & 12磅 & 6磅 &  \\
正文 & * & 小四 & 两端对齐 & 0磅 & 0磅 & \footnotesize{*未注明字体的,统一按“中文宋体,英文、数字Times New Roman”原则} \\
 &  &  & (首行缩进) &  &  &  \\
页眉 &  & 五号 & 居中 & 0磅 & 0磅 &  \\
页码 &  & 小五 & 居中 & 0磅 & 0磅 &  \\
脚注 &  & 小五 & 两端对齐 & 0磅 & 0磅 &  \\
参考文献 &  & 五号 & 两端对齐 & 0磅 & 0磅 &  \\
 &  &  & (悬挂缩进) &  &  &  \\
附录 &  & 五号 & * & 0磅 & 0磅 & \footnotesize{*根据附录形式选择合适的排版方式。} \\
图片 &  & 五号* & 居中 & 6磅 & 0磅 & \footnotesize{*图中文字显示大小跟图题文字一致。} \\
图题 &  & 五号 & 居中* & 6磅 & 12磅 & \footnotesize{*超过一行的图题并非居中,详见\ref{sec2.4.1}} \\
表格 &  & 五号 & 居中 & 0磅 & 6磅 & \footnotesize{一般采用三线表样式} \\
表题 &  & 五号 & 居中* & 12磅 & 6磅 & \footnotesize{*超过一行的表题并非居中,详见\ref{sec2.4.2}} \\
图表附注 &  & 五号 & 顶格 & 6磅 & 6磅 &  \\
公式 &  & 小四 & 居中 & 6磅 & 6磅 &  \\
公式编号 &  & 小四 & 右对齐* & 6磅 & 6磅 & \footnotesize{*公式编号前不加引导线,详见\ref{sec2.5}} \\
\bottomrule
\end{tabular}
\end{table}

其他要求:

1. 各级标题不得置于页面的最后一行,即须与下段同页;

2. 两个标题之间无正文时,第二个标题的段前距设置为0磅;

3. 图、表、公式统一采用单倍行距;

4. 只有一、两行文字的,不得单独作为一页内容;

5. 除各章最后一页外,中间页面不得出现较大空白;

6. 必要时,可在规定的格式要求基础上适当微调,以利于排版,但显示效果不得与规定的格式要求存在明显差距。

\section{图表}

图、表和表达式应居中放置,按章编号,用两位阿拉伯数字分别编号,前一位数字为章的序号,后一位数字为本章内图、表或表达式的顺序号,两个数字中间用短横线连接,如“图1-4”、“表1-2”,“式(3-5)”。在图、表紧邻的前文中,须有相应提示,例如“如图1-2所示”“见表1-3”等。若图或表中有附注,采用英文小写字母顺序编号,附注写在图或表的下方。

引用文献中的图、表时,除在正文文字中标注参考文献序号以外,还须在\textbf{图题、表题的右上角标注参考文献序号}。

\subsection{图}\label{sec2.4.1}

图要求精练,具有“自明性”,即只看图、图题和图例,不阅读正文,就可理解图意。作图须符合相关标准或行业惯例;图片清晰、易于分辨,能满足复印、微缩需要;图中的术语、符号、单位等应与正文表述中一致。

每个图应有\textbf{图序、图题},五号\textcolor{red}{宋体},单倍行距,居中置于图的下方,段前距6磅,段后距12磅,图序与图题文字之间空1个汉字符的宽度;超过一行时,两端对齐,左右缩进4字符。若有分图,\textbf{分图题置于主图题之后};分图序号用 (a)、(b)、(c)等表示,五号\textcolor{red}{宋体},居中置于对应分图正下方,也可置于对应分图左上角等位置,但应全文统一,如图\ref{fig:有分图的情况的示例}。若有附注,用五号字顶格写在图题下方,首段段前距、末段段后距设为6磅。

\begin{figure}[!ht]  
    \centering
    \begin{subfigure}[t]{0.45\textwidth}
        \centering
        \includegraphics[width=\linewidth]{example-image-a.pdf}
        \caption{}
    \end{subfigure}
    \quad
    \begin{subfigure}[t]{0.45\textwidth}
        \centering
        \includegraphics[width=\linewidth]{example-image-b.pdf}
        \caption{}
    \end{subfigure}
    \caption{分图题置于主图题之后:(a)分图一的标题;(b)分图二的标题}
    \label{fig:有分图的情况的示例}
\end{figure}

\textbf{图中文字显示大小应与图题文字大小一致}。若非直接引用的图,除缩略词、单位外,图中坐标轴、说明性文字等应\textbf{统一使用中文}。

\textbf{图和图题须编排在同页,图题不得跨页}。如图太宽,可逆时针方向旋转90°放置。图页面积太大时,可分别配置在两页上,次页需注明“续”字,并注明图题,如“续图1-1 高光谱遥感技术成像示意图”。如需英文图名,应中英文对照,英文序号与内容应与中文一致,在中文下方另起一行。

\subsection{表}\label{sec2.4.2}

表应有“自明性”。表中参数应标明量和单位的符号。为使表格简洁明了,\textcolor{red}{均}采用\textbf{三线表}(必要时可加辅助线,三线表无法清晰表达时可采用其他格式),表的上下边线为单直线,线粗1.5磅,第三条线为单直线,线粗1磅。

每个表应有\textbf{表序、表题},五号\textcolor{red}{宋体},单倍行距,居中置于表的上方,段前距12磅,段后距6磅;表序与表题文字之间空1个汉字符的宽度;超过一行时,两端对齐,左右缩进4字符。表题的表格之后首段正文的段前距6磅。若有附注,用五号字顶格写在表下方,首段段前距、末段段后距设为6磅。

表单元格中的文字一般应居中书写(上下居中、左右居中),不宜左右居中书写的,可采用两端对齐的方式书写;五号宋体、单倍行距、段前空3磅,段后空3磅。

\textbf{表格一般不跨页编排},仅当一页内编排不下时才可转页,以续表形式接排,格式同前,续表均应重复表头,并在每页表序前加“续”字,如“续表2-3 不同主成分分量取值(K)对两个模型分类精度的影响情况(\%)”。

\section{公式}\label{sec2.5}

公式主要是指数学表达式,例如数学公式,也包括文字表达式。公式一般另起一行居中书写,采用与正文相同的字号,或另起一段空两个字符书写,全文须使用统一采用某一种格式。公式应有序号,序号加括号置于公式所在行\textbf{最右端,不加任何连线}。在公式紧邻的前文中,须有相应提示,例如“见式(1-2)”等。

较长的表达式必须转行时,应在“=”或者“+”“-”“×”“/”等运算符或者“]”“\}”等括号之后回行。上下行尽可能在“=”处对齐。例如:
\begin{equation}
    \begin{aligned}
    x&=1+2\\
     &=3
    \end{aligned}
\end{equation}

公式采用Cambira Math或Times New Roman字体,\textbf{单倍行距},段前、段后距均为6磅。
当公式不是独立成行书写时,应尽量将其高度降低为一行,例如,将分数线书写成“/”,将根号改为负指数,例如$8^{-1/2}$。

\section{量和单位}

严格执行《量和单位》国家标准(GB 3100-93)、GB/T 3101-93、GB/T 3102.1~13-93等共15项)有关量和单位的规定。计量单位书写,可以采用国际通用符号,也可以用中文名称,但\textbf{全文应统一},不得两种混用;以人名命名的计量单位首字母大写。

\textbf{量的符号采用斜体书写,计量单位用正体书写};量与单位间用斜线隔开,例如:$I/\mathrm{A}$,$\rho/\mathrm{kg \cdot m^{-3}}$,$F/\mathrm{N}$,$v/\mathrm{m \cdot s^{-1}}$等。

不定数字之后允许使用中文计量单位符号,如“几千克”;非中文数值和计量单位之间空\textbf{1个字符},例如“1 m”。表达时刻时应采用中文计量单位,如“上午8点1刻”,不能写成“8h15min”。

\section{标点符号和数字}

执行国家标准《标点符号用法》(GB/T15834-2011)和《出版物上数字用法》(GB/T 15835-2011)相关规定。除习惯用中文数字表示的以外,一般数字统一用阿拉伯数字。

\section{脚注}

脚注是对论文中某一特定内容所做的进一步解释或补充说明,相关内容切忌直接在文中注释,一般放在该页地脚。脚注序号用阿拉伯数字加圆圈按页编排,每页的脚注序号均从“\textcircled{1}”开始,采用非“上标”样式,与脚注内容之间空半个汉字符。脚注用小五号字,单倍行距,两端对齐,悬挂缩进1.5字符。

\section{参考文献}\label{sec2.9}

参考文献应具有权威性和时效性,列示须实事求是,\textbf{引用过的文献必须著录,未引用的文献不得虚列}。文献标注及书写格式执行国家标准《信息与文献 参考文献著录规则》(GB/T 7714-2015)相关规定。

\subsection{文献引用标注}

1. 采用顺序编码制,按正文中引用文献出现的先后顺序连续编码,并将序号置于方括号“[ ]”中,以上标方式标注在引用位置;

2. 同一处引用多篇文献,应将各篇文献的序号在方括号“[ ]”中全部列出,各序号间用逗号,如遇连续序号,起讫序号间用短横线“-”连接,例如“通过提取地物目标可分性更好的空谱特征来实现其有效分类[65, 66, 70-77]”;

3. 作为句子有效成分的引用标志不用上标,如“由文献[4, 7-10]可知”;

4. 重复引用同一文献,始终标注第一次引用的序号;

5. 除引用的图表标题外,不得将引用文献引用标注置于各级章节标题处。

\subsection{文献书写格式}

参考文献使用\textbf{五号字}。常见参考文献书写格式如表\ref{tab2-3}所示。


\begin{table}[!ht]
\centering
\caption{常见参考文献书写格式}
\label{tab2-3}
\begin{tabular}{m{4em}<{\centering}m{0.85\textwidth}<{\raggedright}}
\toprule
文献类型 & 书写格式 \\
\midrule
期刊论文 & [序号] 作者. 文章题目[J]. 期刊名, 年, 卷(期): 起-止页码. \\
会议论文 & [序号] 作者. 题名[C]. 会议名, 会议地, 会议年: 起-止页码. \\
专著 & [序号] 作者. 书名[M]. 译者. 版本. 出版地: 出版者, 出版年: 起-止页码. \\
学位论文 & [序号] 作者. 题名[D]. 授位单位所在地: 授位单位, 授位年: \textcolor{red}{起-止页码}. \\
报纸文章 & [序号] 作者. 题名[N]. 报纸名, 出版日期 (版面数). \\
报告 & [序号] 作者. 题名[R]. 出版地: 出版者, 出版年.\\
专利 & \hangindent=3em [序号] 专利申请者或所有人. 专利题名[P]. 专利国别, 专利号. 公告日期或公开日期. \\
标准 & [序号] 发布单位. 标准名: 标准号[S]. 出版地: 出版者, 出版年: 起-止页码. \\
电子文献 & \hangindent=3em [序号] 作者. 题名[文献类型标识/文献载体标识]. 出版地: 出版者, 出版年: 起-止页码 (更新或修改日期) [引用日期]. 获取或访问路径. 数字对象唯一标识符.\\
\bottomrule
\end{tabular}
\end{table}

说明:

1. 参考文献\textbf{不跨页编排},即一条文献所在段中不分页;

2. 参考文献\textbf{悬挂缩进、两端对齐},所有\textbf{文献编号左侧对齐},文献编号和文献内容之间统一\textcolor{red}{空}2.5\textcolor{red}{个字符(模版已设置)},所有\textbf{文献内容左侧对齐};

3. 作者\textbf{姓在前、名在后,英文姓全拼、首字母大写,英文名大写缩写且不加点},例如“Harrington R F”(Roger F. Harrington),“Li M”(Li Moumou);

4. 作者姓名之间用逗号隔开,\textbf{最多写到第3位作者},余者用“, 等”或“, et al.”代替;

5. 除特殊名词外,\textbf{英文文献标题(论文题目、书名)仅第一个单词的首字母大写},其余全部小写;\textbf{英文文献出处(期刊名、会议名等)一般每个单词的首字母大写},只有长度为1~4字母的虚词全部小写,例如“in”“with”;

6. 日期统一用8位数字“YYYY-MM-DD”格式,年、月、日用短横线隔开;

7. 若文献本身不具备个别著录要素,则不著录该要素及对应的标识符号,例如,没有期号的期刊论文,其格式书写为“[序号] 作者. 文题[J]. 期刊名, 年, 卷: 起-止页码.”;

8. 初版的专著不著录版本,电子文献数字对象唯一识别符仅在获取或访问路径中不含数字对象唯一识别符时著录。

常见文献类型及标识代码见表\ref{tab2-4},电子资源载体及标识代码见表\ref{tab2-5}。\textbf{参考文献实例见第\textcolor{red}{20}页}。


\begin{table}[!ht]
\centering
\caption{文献类型和标识代码}
\label{tab2-4}
\belowrulesep=0pt % 三线表的中间的辅助线
\aboverulesep=0pt
\begin{tabularx}{\textwidth}{ 
>{\centering\arraybackslash}X 
>{\centering\arraybackslash}X|
>{\centering\arraybackslash}X
>{\centering\arraybackslash}X
}
\toprule
参考文献类型 & 标识代码 & 参考文献类型 & 标识代码 \\
\midrule
普通图书 & M & 专利 & P \\
会议录 & C & 数据库 & DB \\
汇编 & G & 计算机程序 & CP \\
报纸 & N & 电子公告 & EB \\
期刊 & J & 档案 & A \\
学位论文 & D & 舆图 & CM \\
报告 & R & 数据集 & DS \\
标准 & S & 其他 & Z \\
\bottomrule
\end{tabularx}
\end{table}

(图书例如\cite{杨新文2017轨道交通轮轨噪声机理预测与控制,教育部国家语言文字工作委员会2018通用规范汉字表,霍夫斯基主编1981禽病学,王夫之1865宋论,白书农1998植物开花研究,Itkis1976control});标准例如\cite{全国信息与文献标准化技术委员会2007学位论文编写规};专利例如\cite{姜锡洲1983一种温热外敷药制备方案,王士民2023一种用于研究盾构隧道管片上浮形态的模型实验装置};期刊例如\cite{高晗2020深度学习模型压缩与加速综述,zeng2019spectrum,rathgeb2023extended};学位论文例如\cite{陈梦玉2021基于人工智能算法的无线网络认知通信信号波形技术研究};报纸例如\cite{仲音2023敢做善为抓落实};报告例如\cite{冯西桥1997核反应堆压力容器的LBB分析};自定义例如\cite{萧钰2001出版业信息化驶入快车道}。

\begin{table}[!ht]
\centering
\caption{电子资源载体类型和标识代码}
\label{tab2-5}
\belowrulesep=0pt
\aboverulesep=0pt
\begin{tabularx}{\textwidth}{ 
>{\centering\arraybackslash}X|
>{\centering\arraybackslash}X
}
\toprule
电子资源的载体类型 & 载体类型标识代码 \\
\midrule
磁带(magnetic tape) & MT \\
磁盘(disk) & DK \\
光盘(CD-ROM) & CD \\
联机网络(online) & OL \\
\bottomrule
\end{tabularx}
\end{table}

\section{攻读学位期间取得的研究成果}\label{sec2.10}

攻读学位期间取得的研究成果只列出在攻读博士(硕士)学位期间取得的\textbf{与学位论文内容密切相关、能反映学位论文研究工作}的研究成果,例如发表和已录用的学术论文、专著/译著、参与的科研项目、专利、作品、科研获奖等。书写格式如下:

1. 学术论文书写格式与参考文献基本一致,尚未刊载但已收到正式录用函的学术论文加括号注明已被***期刊录用;

2. 专著/译著尚未出版但已被出版社决定出版的专著/译著加括号注明出版社名称和预计出版时间;

3. 专利书写格式与参考文献基本一致,处于\textcolor{red}{申请阶段}的专利在专利位置填写专利申请号,并加括号注明是专利申请号;

4. 作品大致书写格式:作者. 作品名称. 创作时间. 材料形式. 作品尺寸. 作品地点. 参展信息. 是否获奖等信息;

5. 科研获奖大致书写格式:获奖人. 项目名称. 获奖名称及等级, 发奖机构, 获奖日期;

6. 公开的研究报告书写格式与参考文献基本一致;

7. 未列举的其他类型成果,可参照上述格式要求书写;

8. \textbf{本人姓名加粗},列出所有作者,若作者超过5人,也可按“本人姓名(本人排名次序/总人数)”格式代表所有作者。

\section{页面设置、页眉和页码}

\subsection{页面设置}

博士学位论文和硕士学位论文除“中文封面”、“英文封面”、“学位论文指导小组、公开评阅人、和答辩委员会名单”和“学位论文独创性声明、学位论文使用授权”等四部分采用单面印刷外,从中文摘要开始后面的部分均采用A4幅面白色70克以上80克以下(彩色插图页除外)双面印刷,正文从另页右页开始。页面设置数据如下:

页边距:上—3.0厘米,下—3.0厘米,左—3.0厘米,右—3.0厘米,装订线0厘米;

页码范围:普通

页眉距边界:2厘米,页脚距边界:2厘米。

\subsection{页眉}

论文除中文摘要之前的前置部分(封面,中、英文扉页,独创性声明及论文使用授权页)不编排页眉外,其余页面均须编排页眉。

页眉采用\textbf{五号字}居中书写,中文用宋体,英文和数字用Times New Roman体;页眉线为单横线,线宽0.75磅。

中文摘要及之后的前置部分,页眉为各部分内容的标题,例如:“摘要”“ABSTRACT”“目录”“图目录”“表目录”“主要符号表”“缩略词表”。从第1章第1页开始,至论文最后一页,\textcolor{red}{\textbf{页眉统一用“西南交通大学博/硕士学位论文”}}。

\subsection{页码}

页码位于页面底端居中书写,\textbf{小五号字},Times New Roman体,页码数字两侧不加“-”等修饰线。

中文摘要及之后的\textbf{前置部分}(中、英文摘要,目录,图目录,表目录,主要符号表、缩略词表等注释表),\textbf{用大写罗马数字}“\uppercase\expandafter{\romannumeral1}、\uppercase\expandafter{\romannumeral2}、\uppercase\expandafter{\romannumeral3}......”从 “\uppercase\expandafter{\romannumeral1}”开始连续编排页码。从正文“第1章”(“引言”、绪论)开始,至论文最后一页,用\textbf{阿拉伯数字}“1、2、3......”从“1”开始连续编排页码。

\section{书脊}

\textcolor{red}{书脊信息应包括:获取学位年份、学位等级、学位论文中文题目、学生姓名、西南交通大学三部分(学位论文中文题目、学生姓名请用相应内容覆盖)。}

\textcolor{red}{中文四号黑体,英文、数字、字母等采用三号Times New Roman字体。论文题目距离顶端2厘米。}

\section{本章小结}

本章主要讲述了学位论文的具体格式范式。

\textbf{涉密学位论文的印刷、制作、传递、存档等,须符合国家、学校相关保密要求。} 