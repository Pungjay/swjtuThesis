%-------------------------------------------------------%
%                        第3章内容
%-------------------------------------------------------%
\chapter{印制要求}\label{第3章}

\section{论文装订}

学位论文一律左侧装订。

装订时严格按照以下顺序排列:

\begin{enumerate}
    \item 论文封面
    \item 论文中、英文扉页
    \item 西南交通大学学位论文独创性声明
    \item 西南交通大学学位论文使用授权
    \item 中文摘要
    \item 英文摘要
    \item 目录
    \item 图/表目录(如须)
    \item 缩略词等注释表(如须)
    \item 论文正文(绪论、具体研究内容、结论、致谢、参考文献)
    \item 附录
    \item 攻读学位期间取得的科研成果
\end{enumerate}

\section{页面设置}

学位论文页面设置按照2.11.1具体要求。

\section{单面及双面印刷}

\textbf{中文摘要之前}的前置部分(封面,中、英文扉页,独创性声明及论文使用授权页)采用\textbf{单面印刷}。

\textbf{中文摘要及之后}的前置部分(中、英文摘要,目录,图目录,表目录,主要符号表、缩略词表等注释表)\textbf{采用双面印刷;若其中某部分页数为奇数,则该部分最后一页单面印刷}。例如:若“摘要”只有1页,则其页码是“\uppercase\expandafter{\romannumeral1}”,第“\uppercase\expandafter{\romannumeral1}”页纸的背面为空白(无页眉或页码);“ABSTRACT”用新的一张纸印刷,页码从“\uppercase\expandafter{\romannumeral2}”开始。\footnote{除用于打印的版本外,电子版论文中一律不得出现空白页。论文打印建议使用PDF格式,为方便一次性双面打印,打印时可在PDF文件相应位置(例如只有1页的摘要之后)插入1页空白页。应注意,这些额外添加的空白页均不得编排页眉和页码;论文中出现的页码应前后连续,不得中断。}

\textbf{从第1章第1页开始,至论文最后一页,所有页面均双面印刷},但每一章的首页须在奇数页。

\section{信息填写}

\textbf{除提交盲审的学位论文外},提交的学位论文须按要求将封面、扉页等页面的相关信息\textbf{填写完整}。

纸质版学位论文,导师及研究生本人须在独创性声明和论文使用授权相应位置签字;电子版学位论文,独创性声明及论文使用授权页须为导师和研究生本人签字的扫描页。

\section{本章小节}

本章主要讲述了学位论文的印制要求。