%-------------------------------------------------------%
%                        第4章内容
%-------------------------------------------------------%
\chapter{结论}\label{第4章}

%\section{主要研究结论}

%\section{主要创新点}

%\section{研究展望}

本次编写在上一版基础上,根据最新标准进一步完善了西南交通大学研究生学位论文撰写规范,优化了文档模板,直观展示了学位论文格式,并预置了主要文字样式,以利于同学们写作。本文档已尽最大可能确保所呈现内容的格式符合规范要求,但限于编者水平、软件环境、兼容性等客观因素,不能保证在本文档基础上撰写的学位论文绝对合规,\textbf{应以本文档所陈述的格式规范要求为判断依据}。

发布学位论文撰写范式的主要目的是统一学位论文的最终呈现形式,\textbf{对于要求格式的具体实现方式不作要求}。同学们可根据自身情况,选择合适的文字编辑软件,例如LaTeX、WPS等,遵照本规范所作要求,结合本文档示例,撰写学位论文,确保论文格式规范。

如对本范式要求或本文档编排方式有任何意见或建议,请联系研究生院学位办公室。